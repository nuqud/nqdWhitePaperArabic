نقود: عملة رقمية مشفرة لامركزية




جعفر مساعد
‫‪jafar@mussa.id‬‬
‫‪www.nuqud.org‬‬


‫‪20‬‬ تمّوز ‫‪2020‬‬




‫ملخص‬
مستقبل التمويل لامركزي. حيث تكون المعاملات مباشرة عبر الشّابكة من ند لند بشكل آمن ودون الحاجة إلى طرف ثالث موثوق به. يستمر السوق في النمو، مما يجلب معه المزيد من الحرية. نقود جزء من حركة التمويل اللامركزي. فهي تجمع بين خصائص العملة المستقرة والعملة المشفرة الرقمية. بعرض ثابت مسبوق التعدين مرتبط بحد أدنى للقيمة، بدون حد أقصى. في البداية تعتمد قيمة نقود على قيمة الذهب. ثم ستعتمد كليا على السوق. مع اكتساب نقود للشعبية، تكتسب فائدة كشكل من أشكال العملة الرقمية المشفرة. موجودة كعَقد ذكي على سلسلة الكتل إثريوم، مؤمن بواسطة ملايين أجهزة الحاسوب في جميع أنحاء العالم. تسمح التبادلات اللامركزية بالمعاملات المباشرة عبر الشّابكة للعملات المشفرة من ند لند. سيتم إدراج نقود في العديد من منصات التمويل اللامركزي. تسهل آلية التبادل اللامركزي المعاملات المؤتمتة بين العملات المشفرة على سلسلة الكتل إثريوم. نقود هو مشروع مفتوح المصدر، صدر بترخيص معهد ماساتشوستس للتكنولوجيا. لا تتردد في المساهمة. نحن نشهد ثورة، نحن في بداية عصر ما بعد البتكوين. يستمر عدد الأشخاص الذين يتبنون هذا الإصدار الجديد من التمويل في الازدياد. تقلبات العملة المشفرة في تناقص مستمر. في المستقبل القريب، يمكن استخدامها بشكل كامل في حياتنا اليومية. احجز مقعدك، القمر وجهتنا.


كلمات دالة:  نقود، لامركزي، مشفر، سلسلة الكتل، إثريوم، تمويل، حرية




‫‪1‬‬  ‫مقدمة‬


مكنت ثورة الاتصالات العالمية من ظهور نسخة جديدة متصلة من التمويل التقليدي، حيث استخدمت المصارف والشركات الأخرى تقنية الحاسوب والشبكات للمدفوعات وحفظ السجلات. في هذا الإصدار، تعمل المؤسسات المالية التقليدية كأطراف ثالثة موثوق بها لمعالجة المدفوعات الرقمية. تطبق هذه المؤسسات تكلفة الوساطة، مما يزيد من تكلفة المعاملة، ويحد من الحد الأدنى لحجم المعاملة. يطبقون أيضًا "تاريخ القيمة"، أي معاملة تمت يوم الخميس ستقيد يوم الاثنين. زيادةً على احتمال فقدان المعاملات. يتم قبول طرق معينة للاحتيال على أنها حتمية. يمكن تجنب هذه التكاليف، التأخيرات والشكوك باستخدام العملة المادية. من هنا تأتي الحاجة إلى آلية لتسديد المدفوعات عبر قناة اتصال بدون طرف موثوق به.


تعتبر اللامركزية نقل للسلطة من المركز إلى كيانات مستقلة، مرتبطة ببعضها البعض بواسطة قواعد ند لند (Peer-to-peer). سلسلة الكتل (Blockchain) هو نظام لامركزي، يتكون من قواعد بيانات، تسمى كتل (Block)، موزعة عبر مواقع مادية مختلفة، مرتبطة ببعضها البعض بواسطة سلسلة توقيعات رقمية (Chain)، يحكمها إثبات العمل (إع) ((Proof-of-work (PoW)ا[1]. بتكوين ((Bitcoin (BTC) هي أول عملة رقمية لامركزية يمكن إرسالها من مستخدم إلى مستخدم على شبكة بتكوين من ند لند دون الحاجة إلى وسطاء. تم اختراعها في عام 2008 من قبل شخص مجهول أو مجموعة من الأشخاص باستخدام اسم ساتوشي ناكاموتو. يتم التحقق من المعاملات من خلال وحدات الشبكة المتماثلة، والتي تسمى العُقد (Nodes)، ولكل منها نفس القدرة على الإرسال، الاستلام والحساب مثل العُقد الأخرى في شبكتها، المرتبطة ببعضها البعض باستخدام التشفير. يتم تسجيل جميع المعاملات في دفتر رقمي عام (Ledger) معروف باسم سلسلة الكتل. بعد بتكوين، ظهرت العديد من العملات البديلة (Altcoins). وهي تستند إلى نفس تقنية بتكوين، باستخدام آلية توافق (Consensus) لإنتاج الكتل أو التحقق من صحة المعاملات. تتمثل قيمتها المضافة في توفير إمكانات جديدة أو إضافية، مثل العُقود الذكية (كإثريوم (Ethereum)) أو تقلب الأسعار المنخفض (كالعملات المستقرة (Stablecoins)).


التمويل اللامركزي ((Decentralized finance (DeFi) هو شكل من أشكال التمويل القائم على سلسلة الكتل، يستخدم العُقود الذكية. تسمح منصات التمويل اللامركزي (DeFi) للأشخاص بإقراض أو اقتراض الأموال من الآخرين، وتداول العملات المشفرة، دون الحاجة إلى طرف ثالث موثوق به. سيكون مشروع نقود "التحدي" لنكون جزءًا من سوق التمويل اللامركزي.


‫‪2‬‬  تصميم


نقود (Nuqud) هو مشروع مفتوح المصدر لعملة مشفرة رقمية لامركزية [2]، إعتماداً على إطار العمل اوپن زيپلين (OpenZeppelin)، صادر بترخيص معهد ماساتشوستس للتكنولوجيا (رخصة إم أي تي) (MIT Licence). اوپن زيپلين (OpenZeppelin) هو إطار عمل مفتوح المصدر لبناء عُقود ذكية آمنة. يوفر إطار العمل هذا مجموعة كاملة من منتجات الأمان وخدمات التدقيق لبناء وإدارة وفحص جميع جوانب تطوير البرمجية وعملياتها للتطبيقات اللامركزية. يمنحك ترخيص إم أي تي القدرة على الولوج لمصدر البرمجية، تعديلها وإعادة توزيعها. يؤكد تصميم نقود مفتوح المصدر رغبتنا في الحفاظ على حرية الولوج للمعلومة. جميع المنتجات الأخرى التي سيتم تطويرها كجزء من هذا المشروع ستكون مفتوحة المصدر. لم يتم تطوير بتكوين لإثرائنا، ولكن قبل كل شيء للحفاظ على حريتنا.


2.1  وحدات و قابلية القسمة
الرمز المستخدم للتعبير عن نقود هو ن.ق.د (NQD). شعارها هو حرف النون بشرطتين   . الكميات الصغيرة من نقود المستخدمة كوحدات بديلة هي ميلينقود (mNQD) ونقد (مفرد نقود). النقد هو أصغر جزء في نقود يمثل نقود، واحد على كوينتليون من نقود. ميلينقود يساوي 1/1000 نقود؛ واحد على الألف من نقود أو واحد كوادريليون نقد.


2.2  مفهوم
التعدين هو العملية التي يتم من خلالها جمع معاملات العملة المشفرة والتحقق منها وتسجيلها. يتمثل الدور الأساسي للمُعدنين في الحفاظ على سلامة الشبكة وإدخال عملات جديدة في النظام. هذه العملية مكلفة للغاية من حيث الطاقة. ضد الاحتباس الحراري، من الضروري تقليل استهلاكنا للطاقة، باستخدامها بشكل أكثر فاعلية وتنويع مصادرنا، عن طريق استبدال الوقود الأحفوري بالطاقات المتجددة. يسمح لنا التعدين المسبق للرموز بتخفيض استهلاكنا للطاقة جزئيًا. يتم تعدين نقود مسبقًا بعرض ثابت يبلغ 21،000،000 رموز، مربوطة بحد أدنى 1.00 ميكروغرام من الذهب لكل رمز، بدون حد أقصى للقيمة (شكل 1).




  

شكل 1: إلى القمر!


عندما يكون الطلب أكبر من العرض، قد تزيد القيمة أعلى من قيمة الربط الأولية البالغة 1.00 µغ من الذهب. على سبيل المثال، إذا كان كل شخص يريد 1000 نقود، فإن القيمة السوقية السائدة لنقود هي 1.00 µغ لكل منهما. إذا أراد 2000 شخص لكل منهم 1000 نقود، فإن القيمة السوقية السائدة لنقود ترتفع إلى 2.00 µغ، وبالتالي زيادة القيمة السوقية. بمرور الوقت، قد يصل سعر نقود إلى 1.00 م غ، 1.00 غ أو أعلى.


2.3  خصائص
تتميز نقود بخصائصها الفريدة، حيث يتم الجمع بين العملات المستقرة والعملات الرقمية المشفرة. نبدأ بقيمة ثابتة، ثم نتحرك نحو القمر بلا حدود. تمت كتابة عَقد نقود الذكي بصلدتي (Solidity)ا[3] (جدول 1):


إجمالي العرض
	21،000،000
	اسم
	Nuqud |   نقود
	رمز
	ن.ق.د | NQD
	كسور عشرية
	18
	جدول 1: خصائص العَقد الذكي


صلدتي (Solidity) هي لغة برمجة كائنية التوجّه لكتابة العقود الذكية. يتم استخدامها لتنفيذ العقود الذكية على العديد من سلاسل الكتل و المنصات، أبرزها إثريوم (Ethereum). تهدف البرامج التي جمعتها صلدتي إلى تشغيلها على آلة إثريوم الافتراضية (ا ث ف) (Ethereum Virtual Machine). لا يمكن تغيير خصائص عَقد نقود الذكي، بمجرد نشره على سلسلة الكتل. هذا يضمن المزيد من الثقة بين مستخدمي العَقد.


(أ) عرض ثابت محدود:
لدى نقود عرض ثابت يبلغ إجماليه 21،000،000 رمز. لا يمكن أبدًا إنشاء المزيد، بضمان الخوارزمية. إنها انكماشية بطبيعتها. تبدأ بربط  1.00 µغ من الذهب لكل رمز، وتزداد قيمتها بمرور الوقت مع ندرة. نقود "نادرة" بمعنى أنها متوفرة بأعداد قليلة جداً.


"الندرة هي نقطة البداية الأساسية لجميع الاقتصاديات، وأهم ما تنطوي عليه هو فكرة أن كل شيء له فرصة تثمين."— سيف الدين عموص [4]


القيمة السوقية تساوي سعر الرمز مضروبًا في عدد الرموز القائمة. مع ربط بقيمة 1.00 µغ من الذهب، تبدأ نقود بحد أدنى من القيمة السوقية. بالإضافة إلى الحد الأدنى من القيمة، تعتمد القيمة السوقية الإجمالية كليًا على العرض والطلب. على مدى السنوات الخمس الماضية، كان أداء بعض العملات الرقمية جيدًا للغاية.


لحاملي نقود هناك ارتفاع غير محدود. على عكس العملات المستقرة التقليدية (مرتبطة بـ1 رمز = 1.00 غ من الذهب، مثل دجكس (Digix))، تمتلك نقود عرضًا ثابتًا وبالتالي حد أقصى غير محدود. مع اكتساب نقود للشعبية والتداول، من المحتمل أن تصل القيمة السوقية الإجمالية إلى قيم أعلى من 10 ملايين دولار أو 100 مليون دولار من الناحية النظرية. القمر هو الحد.


(ب) منفعة:
نظرًا لاكتساب نقود للشعبية بين المزيد والمزيد من الناس، فإنها تكتسب الاستخدام والقبول كشكل من أشكال العملة الرقمية المشفرة التي تنتشر بشكل كبير من خلال الكلام الشفهي. يمكن استخدام النقود كمخزن للقيمة ووسيلة للمعاملات. العرض المحدود لنقود يجعلها مثاليةً بشكل خاص كمخزن ذي قيمة مع ارتفاع غير محدود. يمكن نقلها بكميات جزئية (حتى 18 رقمًا عشريًا)، مما يسمح باستخدامها في المعاملات الدقيقة أيضًا.


تحصل العملة المشفرة على فائدتها كوسيلة للدفع بسبب عاملين رئيسيين: (1) تكاليف المعاملات، تكلفة تحويل العملة المشفرة جد متدنية، حيث أن عدد الأطراف هو اثنان فقط من الناحية الفنية؛ (2) وقت المعاملة، هو أكثر مثل لمعاملة نقدية بين شخصين تتم رقميًا. تستخدم سلسلة الكتل تشفير المفتاح العام للسماح بتبادل آمن للقيمة عبر شبكة مفتوحة غير آمنة. يتم تحديد محفظة العملة المشفرة بواسطة المفتاح العام. بعض المحافظ (مثل ميطاماسك (MetaMask)) لا تتطلب التحقق من الهوية. لذلك، فإن سلسلة الكتل قد تسمح بعدم الكشف عن الهوية لمزيد من السرية والحفاظ على الخصوصية.


(ج) أمان:
نقود موجودة كعَقد ذكي ((ERC-20 (Ethereum Request for Comments 20) على سلسلة الكتل إثريوم . كعملة لامركزية مؤمنة بملايين أجهزة الحاسوب في جميع أنحاء العالم. في سلسلة الكتل، نطبق آلية التوافق (Consensus mechanism) مرتين: (1) قبول كتلة (Block) جديدة على سلسلة الكتل. (2) التحقق من موثوقية العُقد، معلومة غير معروفة لدى الأغلبية يتم رفضها. تم تصميم سلسلة الكتل بحيث يمكنها الحفاظ على البيانات المخزنة فيها ويمنع تعديلها.  لأنه عند تخزين جزء من المعلومات في سلسلة الكتل، لا يمكن تعديلها لاحقًا [6] [7]. تعتبر سلسلة الكتل آمنة بحسب التصميم وهي مثال على نظام الحاسوب الموزع مع وجود أخطاء بيزنطية عالية التسامح [8]. وبالتالي، فإن سلسلة الكتل تسمح بنظام توافق لامركزي.


تساعد آلية التوافق في نظام التمويل المشفر أيضًا في منع حدوث حالات معينة من الهجمات الاقتصادية. من الناحية النظرية، يمكن للمهاجم تجاوز التوافق بواسطة السيطرة على 51% من الشبكة. تم تصميم آليات التوافق لجعل هذا ال"هجوم 51%" غير ممكن. تم تصميم آليات متعددة لحل هذه المشكلة الأمنية بطرق مختلفة. هناك نوعان من آليات التوافق: (1) إثبات العمل (إع) ((Proof-of-work (PoW)، المستخدمة حاليًا بواسطة بتكوين (Bitcoin)، يتم بواسطة المُعدنيين، الذين يتنافسون لإنشاء كتل جديدة مليئة بالمعاملات المعالجة. الفائزون يتشاركون الكتلة الجديدة مع بقية الشبكة وتكسب بعض العملات المعدنية المسكوكة (minted) حديثًا. يتم الفوز بالسباق من قبل أي جهاز حاسوب يمكنه حل اللغز الرياضي (Puzzle) بأسرع ما يمكن. ينتج بذلك رابط التشفير بين الكتلة الحالية والكتلة السابقة. حل هذا اللغز هو "إثبات العمل". (2) إثبات الحصة (إح) ((Proof-of-stake (PoS)، قامت إثريوم بالترقية إلى نظام التوافق إح (PoS). يتم إثبات الحصة من قِبل المدققين الذين رهنوا إثر (ETH) للمشاركة في النظام. يتم اختيار المدقق بشكل عشوائي لإنشاء كتل جديدة ومشاركتها مع الشبكة وكسب المكافآت. بدلاً من الحاجة إلى القيام بعمليات حسابية مكثفة، فأنت تحتاج ببساطة إلى وضع إثر الخاص بك في الشبكة. هذا هو ما يحفز السلوك الإيجابي للشبكة. يتم الحفاظ على الشبكة آمنة من خلال الحقيقة أنك ستحتاج إلى 51% من قوة الحوسبة للشبكة للاحتيال على السلسلة. هذا يتطلب استثمارات ضخمة في المعدات والطاقة، أنت كذلك من المحتمل أن تنفق أكثر مما حصلت عليه.


2.4  تبادل
تسمح التبادلات اللامركزية ((Decentralized Exchange (DEX) بمعاملات العملات المشفرة المباشرة من ند لند، أن تتم عبر الشّابكة بشكل آمن ودون الحاجة إلى طرف ثالث موثوق به. في نظام التمويل اللامركزي ((Decentralized Finance (DeFi)، بدلاً من دفتر الطلبات التقليدي، نستخدم صانع السوق المؤتمت ((Automated market maker (AMM)، وهو تبادل لامركزي يعتمد على صيغة رياضية لتسعير الأصول. يمكن أن تختلف هذه الصيغة  من نظام لآخر. يستخدم البعض صيغة بسيطة (مثل يونيسواپ (Uniswap))، بينما يستخدم كورڤ (Curve)، بلنصر (Balancer) وغيرهم صيغًا أكثر تعقيدًا. يعملون بتجمعات السيولة (ت س) ((Liquidity Pool (LP)، يتكون كل ت.س من زوج من الرموز (م، ل). على سبيل المثال، يستخدم يونيسواپ مـ  لـ = كـ، حيث مـ هو مقدار رمز واحد في السيولة، و لـ هو مقدار الآخر. في هذه الصيغة، كـ هو ثابت محدد، يعني السيولة الإجمالية للتجمع.


نقود مدرج في يونيسواپ [9]، المنصة تسهل المعاملات المؤتمتة بين رموز العملات المشفرة على سلسلة الكتل إثريوم من خلال استخدام عُقود ذكية. وبعد ذلك سيتم إدراج "نقود" في تبادلات أخرى. ستدرج بعض التبادلات نقود نظرًا لشعبيتها، وسيتطلب البعض الآخر مستوى معينًا من القيمة السوقية. قم بإنشاء محفظة عملات مشفرة (مثل ميطاماسك (MetaMask)). غذي محفظتك بعملات مشفرة (إثريوم (Ethereum)، دولار (USDC) أو غيرها). اتصل بالتبادل الذي اخترته (مثل يونيسواپ (Uniswap)) و تداول مقابل نقود (NQD$). في البداية، نحدد قيمة الربط، باستخدام تجمع السيولة. هذا هو الحد الأدنى للقيمة التي سنشتري بها. يمكن لأي شخص أن يصبح صانع سوق على تبادل، من خلال توفير تجمع سيولة، وكسب رسوم مقابل تزويد السيولة [10]. في أي وقت يمكنك بيع رموز تجمع السيولة (رموز-ت س) (LP-Tokens) الخاصة بك واسترداد استثمارك.


3  خاتمة
في سوق اليوم، تمكن تقنية سلسلة الكتل الأفراد من التداول مباشرة مع بعضهم البعض. دون الحاجة إلى طرف ثالث موثوق به. نحن نشهد ثورة، ستغير سلسلة الكتل عالم المال إلى الأبد. هناك عصر ما قبل بتكوين وما بعد بتكوين (جدول 2).


قبل-بتكوين
	بعد-بتكوين
	مركزي
	لامركزي
	عميل-خادوم
	ند-لند
	طرف ثالث موثوق به
	إثبات العمل
	طباعة
	تعدين
	سجل طلبيات
	صانع السوق المؤتمت
	مضارب
	صانع سوق
	تضخم
	انكماش
	جدول 2: عصر قبل-بتكوين و بعد-بتكوين


ستتغير لغتنا، وسنكون أكثر دراية بالمفردات الجديدة، ولن نتحدث عن نفس الشيء بعد الآن. إذا احتفظنا بأموالنا في شكلها النقدي التقليدي -في أحسن الأحوال- فسوف تحافظ على استقرار قيمتها حتى العام المقبل. ومع ذلك، فإن التضخم خلال العام الحالي سوف يمثل خسارة في القيمة. على سبيل المثال، إذا كان بإمكانك شراء نفس المنتجات في سوق اليوم مقابل 100$، بعد عام، إذا كان متوسط ​​قيمة التضخم 3%، فإن نفس مجموعة المنتجات تساوي 3% أكثر. بعبارة أخرى، نحتاج إلى 103$ لشراء نفس الشيء، مما يعني أن 100$ أمس أصبحت 97$ اليوم. وبالتالي، فإن التضخم يمثل خسارة في القيمة. قيمة التضخم تعتمد على كل سوق. لنختر حالة الولايات المتحدة [11].


  

الشكل 2: تضخم تراكميّ


"نمر من ذيله: إلى متى يمكن أن يستمر هذا التضخم؟ إذا تم تحرير النمر (التضخم) فسوف يأكلنا؛ ومع ذلك، إذا ركض أسرع وأسرع بينما تمسكنا بشدة، فنحن انتهينا! أنا سعيد لأنني لن أكون هنا لأرى النتيجة النهائية". — فريدريك حايك [12]


بالنسبة لقيم التضخم بين 0.12% و 3.14%، من 2010 إلى 2020، تبلغ قيمة التضخم التراكميّ 18.88%. على سبيل المثال، الشخص الذي احتفظ بـ100$ في حسابه في عام 2010، فإن 100$ تساوي 81.12$ في عام 2020، في نفس السوق. لنفترض أن هذا الشخص نفسه اختار الاحتفاظ بمبلغ 100$ في محفظة البتكوين الخاصة به، بحلول عام 2020، تجاوزت ثروته مليار دولار. نعم، هذه حالة متبني مبكر ولن يكون الأمر كذلك دائمًا، ولكن ستتصرف بتكوين والعملات البديلة مثل أي سلعة أخرى في السوق؛ لا مزيد من التضخم، فقط الانكماش.


يضمن هذا الإصدار الجديد من التمويل مزيدًا من الحرية، الأمان، التدقيق وإخفاء الهوية. تقلبات العملة المشفرة في تناقص مستمر. يبلغ تقلب (Volatility) عملة البتكوين اليوم حوالي 70% (شكل 3)، مقابل 5% للدولار الأمريكي [13].


  

شكل 3: تقلب بتكوين


حتى لو في المستقبل اقتربت القيمة من تقلب العملات المستقرة، فإن قيمة العملة المشفرة ستتطور مثل جميع السلع الأخرى في السوق. في هذه الحالة، سيكون تقلب العملة المشفرة مساويًا لمتوسط ​​التقلب لعملة مستقرة بالإضافة إلى متوسط ​​قيمة التضخم. "أي شخص يمتلك بتكوين يحقق درجة من الحرية الاقتصادية لم تكن ممكنة قبل اختراعها. يمكن لحاملي بتكوين إرسال كميات كبيرة من القيمة عبر الكوكب دون الحاجة إلى طلب الإذن من أي شخص. لا تعتمد قيمة بتكوين على أي شيء مادي في أي مكان في العالم، وبالتالي لا يمكن أبدًا إعاقة، تدمير أو مصادرتها من أي القوى المادية للعالم السياسي أو الإجرامي". — سيف الدين عموص [4]


في المستقبل القريب، سيتم استخدام العملات المشفرة بالكامل في حياتنا اليومية. ومن هنا تأتي الحاجة إلى مشاريع سلسلة الكتل مختلفة لتلبية احتياجاتنا وضمان حريتنا. نقود هو مشروع مفتوح المصدر [2]، بموجب ترخيص معهد ماساتشوستس للتكنولوجيا (رخصة إم أي تي). لا تتردد في المساهمة. نحن في بداية هذا المشروع. بعد ذلك، سنقوم بتطوير منتجات أخرى لتسهيل استخدام العملات المشفرة. من الأسهل بكثير التنقل باستخدام مفتاح خاص من نقود مقارنة مع كنز من الذهب، ومن الأسهل بكثير إرسالها عبر العالم دون الاضطرار إلى المخاطرة بالسرقة أو المصادرة. سوق المستقبل هو سوق لامركزي بالكامل. نحن ننتقل إلى القمر.


مراجع


[1] س. ناكاموتو، "بيتكوين: نظام نقدي إلكتروني من ند لند" (Bitcoin: A peer to-peer electronic cash system)، عنوان: https://bitcoin.org/bitcoin.pdf (زير: 2015.05.15)، 2008.


[2] "مستودع نقود"، عنوان: https://github.com/nuqud (زير: 2021.07.21).


[3] "عَقد نقود الذكي"، عنوان:
https://etherscan.io/token/0x395b73298780323cffdb92576ecb072b3616751a 
(زير: 2021.07.21).


[4] س. عموص، "معيار البتكوين: البديل اللامركزي للبنوك المركزية (The bitcoin standard: the decentralized alternative to central banking)"، جون وايلي وأولاده، 2018.


[5] ج. أوليڤا، أ. حسن و ز. جيانغ، "دراسة استكشافية للعقود الذكية في منصة سلسلة الكتل إثريوم (An exploratory study of smart contracts in the ethereum blockchain platform)،" امپريكال سفتوار انجينرين (Empirical Software Engineering)، م. 25، ر. 3، ص. 1864–1904، 2020.


[6] ت. زيبين، س. شيه، هـ. داي، ش. تشن و هـ. وانغ، "نظرة عامة على تقنية سلسلة الكتل: هندسة، توافق واتجاهات مستقبلية، (An overview of blockchain technology: Architecture, consensus, and future trends)"، في المؤتمر الدولي إي ترپل أي (IEEE) لعام 2017 حول البيانات الضخمة (BigData congress)، ص 557-564، 2017، إي ترپل أي (IEEE).


[7] پ. تسانكوف، أ. دان، د. دراشسلر-كوهين، أ. ڭيرڤي، ف. بوينزلي و م. ڤيشيڤ، "أمان: تحليل أمني عملي للعقود الذكية (Securify: Practical security analysis of smart contracts)،" في مؤتمر حول أمن الحاسوب والاتصالات (ACM SIGSAC 2018)، ص 67-82، 2018.


[8] ر. كيلمان، "الحوسبة المتسامحة في الأتمتة الصناعية (Fault tolerant computing in industrial automation)،" مركز أبحاث إي بي بي (ABB)، ا2005.


[9]  "يونيسواپ. (Uniswap)"،اhttps://https//app.uniswap.org. (زير: 2020.06.20).


[10] "تجمعات السيولة في التمويل اللامركزي"،
https://academy.binance.com/en/articles/what-are-liquidity-pools-in-defi.
(زير: 2021.07.21).


[11] "التضخم السنوي بالولايات المتحدة الأمريكية ".
 https://www.statista.com/statistics/244983/projected. (زير: 2021.07.21).


[12] ف. حايك، نمر من الذيل ( A Tiger by the Tail)، م 4، معهد لودفيغ فون ميزس، 1971.


[13] "التقلب السنوي للدولار الأمريكي." https://vlab.stern.nyu.edu/analysis. (زير: 2020.06.20).